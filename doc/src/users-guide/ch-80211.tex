\chapter{The 802.11 Model}
\label{cha:80211}

% https://summit.omnetpp.org/archive/2016/assets/pdf/OMNET-2016-Session_9-02-Presentation.pdf

\section{Overview}
\label{sec:80211:overview}

IEEE 802.11 a.k.a. WiFi is the most widely used and universal wireless networking
standard. Specifications are updated every few years, adding more features and
ever increasing bit rates. 

In INET, nodes become WiFi-enabled by adding a \nedtype{Ieee80211Interface} to
them. (As mentioned earlier, \nedtype{WirelessHost} and \nedtype{AdhocHost} already
contain one in their default configuration.) APs are represented with the
\nedtype{AccessPoint} node type. WiFi networks require a matching transmission
medium module to be present in the network, which is usually a 
\nedtype{Ieee80211ScalarRadioMedium}.

Operation mode (infrastructure vs ad hoc) is determined by the
ingredients of the wireless interface. \nedtype{Ieee80211Interface} has
the following submodules (incomplete list): 

\begin{enumerate}
  \item \textit{management}: performs association/disassociation with access points, channel scanning, beaconing
  \item \textit{agent}: initiates actions such as channel scanning and connecting to and disconnecting from access points
  \item \textit{MAC}: transmits and receives frames according to the IEEE 802.11 medium access procedure
  \item \textit{PHY}: represents the radio
\end{enumerate}

The management component has several implementations which differ in their role
and level of detail:

\begin{itemize}
  \item \nedtype{Ieee80211MgmtAdhoc}: for ad hoc mode stations
  \item \nedtype{Ieee80211MgmtSta}, \nedtype{Ieee80211MgmtStaSimplified}: for infrastructure mode stations
  \item \nedtype{Ieee80211MgmtAp}, \nedtype{Ieee80211MgmtApSimplified}: for access points
\end{itemize}

The ``simplified'' ones assume that stations are statically associated to an
access point for the entire duration of the simulation (the
scan-authenticate-associate process is not simulated), so they cannot be used
e.g. in experiments involving handover.

\nedtype{Ieee80211MgmtSta} is does not take any action by itself, it requires an agent
(\nedtype{Ieee80211AgentSta} or a custom one) to initiate actions.


The following sections examine the above components.

\section{MAC}
\label{sec:80211:mac}

The \nedtype{Ieee80211Mac} module type represents the IEEE 802.11 MAC.
The implementation is entirely based on the standard IEEE 802.11-2012 Part 11:
Wireless LAN Medium Access Control (MAC) and Physical Layer (PHY)
Specifications.

\nedtype{Ieee80211Mac} performs transmission of frames according
to the CSMA/CA protocol. It receives data and management frames from
the upper layers, and transmits them.

The \nedtype{Ieee80211Mac} was designed to be modular to facilitate experimenting
with new policies, features and algorithms within the MAC layer. Users can
easily replace individual components with their own implementations. Policies,
which most likely to be experimented with, are extracted into their own modules.

The model has the following replaceable built-in policies:

\begin{itemize}
  \item ACK policy
  \item RTS/CTS policy
  \item Originator and recipient block ACK agreement policies
  \item MSDU aggregation policy
  \item Fragmentation policy
\end{itemize}

The new model also separates the following components:

\begin{itemize}
  \item Coordination functions
  \item Channel access methods
  \item MAC data services
  \item Aggregation and deaggregation
  \item Fragmentation and defragmentation
  \item Block ACK agreements and reordering
  \item Frame exchange sequences
  \item Duplicate removal
  \item Rate selection
  \item Rate control
  \item Protection mechanisms
  \item Recovery procedure
  \item Contention
  \item Frame queues
  \item TX/RX
\end{itemize}

\section{Physical Layer}
\label{sec:80211:physical-layer}

\textit{The physical layer} modules (\nedtype{Ieee80211Radio}) deal with modelling
transmission and reception of frames. They model the characteristics of
the radio channel, and determine if a frame was received correctly
(that is, it did not suffer bit errors due to low signal power or
interference in the radio channel). Frames received correctly are passed
up to the MAC.

On the physical layer, one can choose from several radios with different levels
of detail. The various radio types (with the matching transmission medium types
in parentheses) are:

\begin{itemize}
  \item \nedtype{Ieee80211ScalarRadio} (\nedtype{Ieee80211ScalarRadioMedium}) 
  \item \nedtype{Ieee80211DimensionalRadio} (\nedtype{Ieee80211DimensionalRadioMedium}) 
  \item \nedtype{Ieee80211UnitDiskRadio} (\nedtype{Ieee80211UnitDiskMedium})
\end{itemize}

\section{Management}
\label{sec:80211:management}

\textit{The management layer} exchanges management frames via the MAC with its peer
management entities in other STAs and APs. Beacon, Probe Request/Response,
Authentication, Association Request/Response etc frames are generated
and interpreted by management entities, and transmitted/received via
the MAC layer. During scanning, it is the management entity that periodically
switches channels, and collects information from received beacons and
probe responses.

The management layer has several implementations which differ in their role
(STA/AP/ad-hoc) and level of detail: \nedtype{Ieee80211MgmtAdhoc},
\nedtype{Ieee80211MgmtAp}, \nedtype{Ieee80211MgmtApSimplified}, \nedtype{Ieee80211MgmtSta},
\nedtype{Ieee80211MgmtStaSimplified}. The ..Simplified ones differ from the others
in that they do not model the scan-authenticate-associate process,
so they cannot be used in experiments involving handover.

\section{Agent}
\label{sec:80211:agent}

The agent is what instructs the management layer to perform
scanning, authentication and association. The management layer itself
just carries out these commands by performing the scanning, authentication
and association procedures, and reports back the results to the agent.

The agent component is currently only needed with the \nedtype{Ieee80211MgmtSta}
module. The managament entities in other NIC variants do not have as much
freedom as to need an agent to control them.

\nedtype{Ieee80211MgmtSta} requires a \nedtype{Ieee80211AgentSta} or a custom agent.
By modifying or replacing the agent, one can alter the dynamic behaviour
of STAs in the network, for example implement different handover strategies.


%%% Local Variables:
%%% mode: latex
%%% TeX-master: "usman"
%%% End:

