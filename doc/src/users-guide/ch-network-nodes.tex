\chapter{Network Nodes}
\label{cha:network-nodes}

% TODO: automatic wired interface creation, multi wireless interface, dual stack, forwarding, SSID, manet routing, internet routing protocols, alternative mobility models, L2/L3 devices

% TODO describe how INET models network nodes as modules: routers, hosts,
% switches, mobile devices, access points, sensors, etc., how network nodes are
% composed of smaller components modelling various hardware and software components,
% different aspects such as behaviours, etc.

% TODO describe how INET allows configuring alternative applications, routing protocols,
% network protocols, disable/enable protocols, etc.}

\section{Overview}

Hosts, routers, switches, access points, mobile phones, and other network
devices are all represented by OMNeT++ compound modules. These modules are
assembled from other modules which represent applications, communication
protocols, network interfaces, routing tables, mobility models, energy
models, and other functional units. Many of these modules, including most
of the communication protocols, are simple modules, implemented in C++.
However, some protocols are complicated enough to be compound modules
themselves. They are further divided into smaller functional parts such as
queues, protocol specific data storages, and protocol specific
sub-services.

OMNeT++ connections are used within network nodes to represent
communication opportunities between protocols. These connections are
also facilitate  visualizing what is happening inside network nodes in
the runtime user interface. Packets and messages sent through the connections
represent some software or hardware activity.

\section{Overview 2}

The \ttt{inet.node} package contains various pre-assembled host, router,
switch, access point, and other modules, for example
\nedtype{StandardHost}, \nedtype{Router} and \nedtype{EtherSwitch} and
\nedtype{AccessPoint}. These compound modules contain some customization
options via parametric submodule types, but they are not meant to be
universal, so it may be necessary to create your own node models for
your particular simulation scenarios.

Network interfaces (Ethernet, IEEE 802.11, etc) are usually compound modules
themselves, and are being composed of a queue, a MAC, and possibly other
simple modules. See \nedtype{EthernetInterface} as an example.

Not all modules implement protocols. There are modules which hold data (for
example \nedtype{Ipv4RoutingTable}), perform autoconfiguration of a network
(\nedtype{Ipv4NetworkConfigurator}), move a mobile node around (for example
\nedtype{ConstSpeedMobility}), and perform housekeeping associated with
radio signals in wireless simulations (\nedtype{RadioMedium}).


\section{Ingredients}

All network nodes in the INET Framework are OMNeT++ compound modules that are
mainly composed of the following components:

\begin{itemize}
  \item \emph{Applications} often model the user behavior as well as the
     application program (e.g., browser), and the application layer protocol
     (e.g., \protocol{HTTP}). Applications typically use transport layer
     protocols (e.g., \protocol{TCP} and/or \protocol{UDP}), but they may
     also directly use lower layer protocols (e.g., \protocol{IP} or
     \protocol{Ethernet}) via sockets.
  \item \emph{Routing protocols} are provided as separate modules:
     \protocol{OSPF}, \protocol{BGP}, or \protocol{AODV} for MANET routing.
     These modules use \protocol{TCP}, \protocol{UDP}, and \protocol{IPv4},
     and manipulate routes in the \nedtype{Ipv4\-RoutingTable} module.
  \item \emph{Transport layer protocols} are connected to applications and
     network layer protocols. They are most often represented by simple
     modules, currently \protocol{TCP}, \protocol{UDP}, and \protocol{SCTP}
     are supported. \protocol{TCP} has several implementations: \nedtype{Tcp}
     is the OMNeT++ native implementation; \nedtype{TcpLwip} module wraps the
     lwIP \protocol{TCP} stack; and \nedtype{TcpNsc} module wraps the
     Network Simulation Cradle library.
  \item \emph{Network layer protocols} are connected to transport layer
     protocols and network interfaces. They are usually modeled as compound
     modules: \nedtype{Ipv4NetworkLayer} for \protocol{IPv4}, and
     \nedtype{Ipv6NetworkLayer} for \protocol{IPv6}. The \nedtype{Ipv4NetworkLayer}
     module contains several protocol modules: \nedtype{Ipv4}, \nedtype{Arp},
     and \nedtype{Icmpv4}.
  \item \emph{Network interfaces} are represented by compound modules
     which are connected to the network layer protocols and other network
     interfaces in the wired case. They are often modeled as compound modules
     containing separate modules for queues, classifiers, MAC, and PHY protocols.
  \item \emph{Link layer protocols} are usually simple modules sitting
     in network interface modules. Some protocols, for example
     \protocol{IEEE 802.11 MAC}, are modeled as a compound module themselves
     due to the complexity of the protocol.
  \item \emph{Physical layer protocols} are compound modules also being part
     of network interface modules.
  \item \emph{Interface table} maintains the set of network interfaces
     (e.g. \texttt{eth0}, \texttt{wlan0}) in the network node. Interfaces
     are registered dynamically during initialization of network interfaces.
  \item \emph{Routing tables} maintain the list of routes for the corresponding
     network protocol (e.g., \nedtype{Ipv4RoutingTable} for \nedtype{Ipv4}).
     Routes are added by automatic network configurators or routing protocols.
     Network protocols use the routing tables to find out the best matching
     route for datagrams.
  \item \emph{Mobility modules} are responsible for moving around the network
     node in the simulated playground. The mobility model is mandatory for
     wireless simulations even if the network node is stationary. The mobility
     module stores the location of the network node which is needed to compute
     wireless propagation and path loss. Different mobility models are provided
     as different modules. Network nodes define their mobility submodule with
     a parametric type, so the mobility model can be changed in the configuration.
  \item \emph{Energy modules} model energy storage mechanisms, energy
     consumption of devices and software processes, energy generation of devices,
     and energy management processes which shutdown and startup network nodes.
  \item \emph{Other modules} with particular functionality such as
     \nedtype{PcapRecorder} are also available.
\end{itemize}

\section{Node Architecture}

TODO describe how INET protocols are connected to each other using the built-in message dispatching mechanism.

\section{Customization}

The built-in network nodes are written to be as versatile and customizable
as possible. There are several mechanisms that make such flexibility possible:

The simplest way is the use of \textit{gate vectors and submodule vectors}. The
sizes of vectors may come from parameters or derived by the number of
external connections to the network node. For example, one can have an
\protocol{Ethernet} switch that has as many ports as needed, i.e. equal to
the number of \protocol{Ethernet} devices connected to it.

TODO example

Another often used way of customization is \textit{parametric types}, that is, the
type of a submodule (or a channel) may be specified as a string parameter.
For example, the relay unit inside an \protocol{Ethernet} switch has
several alternative implementations, each one being a distinct module type.
The switch model contains a parameter which allows the user to select the
appropriate relay unit implementation.

TODO example

The most flexible way of customizing modules is \textit{inheritance}: a derived
module may add new parameters, gates, submodules or connections, and may
set inherited unassigned parameters to specific values.

TODO example


\section{[Applications]} % needed?
\label{subsec:applications}

TODO describe how INET models running applications as modules: ping, connectionless traffic, connection oriented traffic, voip, video, etc.

\section{[Protocols]} % needed?
\label{subsec:protocols}

TODO describe how INET models data and control plane protocols as modules in general

\section{[Routing Protocols]} % needed?

TODO briefly describe how INET models routing protocols: BGP, OSPF, RIP, AODV, etc.

\section{Custom Network Nodes}

Despite the many pre-assembled network nodes and the several available
customization options, sometimes it's just easier to build a network node
from scratch. The following example shows how easy it is to build a simple
network node.

This network node already contains a configurable application and several
standard protocols. It also demonstrates how to use the packet dispatching
mechanism which is required to connect multiple protocols in a many-to-many
relationship.

\nedsnippet{NetworkNodeExample}{Network node example}



%%% Local Variables:
%%% mode: latex
%%% TeX-master: "usman"
%%% End:

