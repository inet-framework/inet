\chapter{The UDP Model}
\label{cha:udp}


\section{Overview}

The UDP protocol is a very simple datagram transport protocol, which
basically makes the services of the network layer available to the applications.
It performs packet multiplexing and demultiplexing to ports and some basic
error detection only.

The frame format as described in RFC768:

\begin{center}
\begin{bytefield}{32}
\bitheader{0,7,8,15,16,23,24,31} \\
\bitbox{16}{Source Port} &
\bitbox{16}{Destination Port} \\
\bitbox{16}{Length} &
\bitbox{16}{Checksum} \\
\wordbox{3}{Data}
\end{bytefield}
\end{center}

The ports represents the communication end points that are allocated by the
applications that want to send or receive the datagrams. The ``Data'' field
is the encapsulated application data, the ``Length'' and ``Checksum'' fields
are computed from the data.

The INET framework contains an \nedtype{UDP} module that performs the encapsulation/decapsulation
of user packets, an \nedtype{UDPSocket} class that provides the application the usual
socket interface, and several sample applications.

These components implement the following statndards:
\begin{itemize}
\item RFC768: User Datagram Protocol
\item RFC1122: Requirements for Internet Hosts -- Communication Layers
\end{itemize}

\section{The UDP module}

The UDP protocol is implemented by the \nedtype{UDP} simple module.
There is a module interface (\nedtype{IUDP}) that defines the gates of the
\nedtype{UDP} component. In the \nedtype{StandardHost} node, the UDP component
can be any module implementing that interface.

Each UDP module has gates to connect to the IPv4 and IPv6 network layer
(ipIn/ipOut and ipv6In/ipv6Out), and a gate array to connect to the applications
(appIn/appOut).

The UDP module can be connected to several applications, and each application
can use several sockets to send and receive UDP datagrams.


\section{UDP applications}

All UDP applications should be derived from the \nedtype{IUDPApp} module interface,
so that the application of \nedtype{StandardHost} could be configured without changing its NED file.

The following applications are implemented in INET:
\begin{itemize}
\item \nedtype{UDPBasicApp} sends UDP packets to a given IP address at a given interval
\item \nedtype{UDPBasicBurst} sends UDP packets to the given IP address(es) in bursts, or acts as a packet sink.
\item \nedtype{UDPEchoApp} similar to \nedtype{UDPBasicApp}, but it sends back the packet after reception
\item \nedtype{UDPSink} consumes and prints packets received from the \nedtype{UDP} module
\item \nedtype{UDPVideoStreamCli},\nedtype{UDPVideoStreamSvr} simulates UDP streaming
\end{itemize}

The next sections describe these applications in details.

\subsection{UDPBasicApp}

The \nedtype{UDPBasicApp} sends UDP packets to a the IP addresses given in the
\fpar{destAddresses} parameter. The application sends a message to one of the
targets in each \fpar{sendInterval} interval. The interval between message and
the message length can be given as a random variable. Before the packet is
sent, it is emitted in the \fsignal{sentPk} signal.

The application simply prints the received UDP datagrams. The \fsignal{rcvdPk}
signal can be used to detect the received packets.

The number of sent and received messages are saved as scalars at the end of the
simulation.

% could be a simple packet generator without the ability to receive packets?

\subsection{UDPSink}

This module binds an UDP socket to a given local port, and prints the
source and destination and the length of each received packet.

% TODO does not accept broadcast messages

\subsection{UDPEchoApp}

Similar to \nedtype{UDPBasicApp}, but it sends back the packet after reception.
It accepts only packets with \msgtype{UDPEchoAppMsg} type, i.e. packets that
are generated by another \nedtype{UDPEchoApp}.

When an echo response received, it emits an \fsignal{roundTripTime} signal.

\subsection{UDPVideoStreamCli}

This module is a video streaming client. It send one ``video streaming request'' to
the server at time \fpar{startTime} and receives stream from \nedtype{UDPVideoStreamSvr}.

The received packets are emitted by the \fsignal{rcvdPk} signal.

\subsection{UDPVideoStreamSvr}

This is the video stream server to be used with \nedtype{UDPVideoStreamCli}.

The server will wait for incoming "video streaming requests".
When a request arrives, it draws a random video stream size
using the \fpar{videoSize} parameter, and starts streaming to the client.
During streaming, it will send UDP packets of size \fpar{packetLen} at every
\fpar{sendInterval}, until \fpar{videoSize} is reached. The parameters \fpar{packetLen}
and \fpar{sendInterval} can be set to constant values to create CBR traffic,
or to random values (e.g. sendInterval=uniform(1e-6, 1.01e-6)) to
accomodate jitter.

The server can serve several clients, and several streams per client.

% FIXME why streamVector? VideoStreamData could be deleted immediately after last byte sent
% TODO this is video-on-demand, support multicast/broadcast video streaming too

\subsection{UDPBasicBurst}

Sends UDP packets to the given IP address(es) in bursts, or acts as a
packet sink. Compatible with both IPv4 and IPv6.

\subsubsection*{Addressing}

The \fpar{destAddresses} parameter can contain zero, one or more destination
addresses, separated by spaces. If there is no destination address given,
the module will act as packet sink. If there are more than one addresses,
one of them is randomly chosen, either for the whole simulation run,
or for each burst, or for each packet, depending on the value of the
\fpar{chooseDestAddrMode} parameter. The \fpar{destAddrRNG} parameter controls which
(local) RNG is used for randomized address selection.
The own addresses will be ignored.

An address may be given in the dotted decimal notation, or with the module
name. (The \cppclass{L3AddressResolver} class is used to resolve the address.)
You can use the "Broadcast" string as address for sending broadcast messages.

INET also defines several NED functions that can be useful:
\begin{itemize}
\item[-] moduleListByPath("pattern",...): \\
         Returns a space-separated list of the modulenames.
         All modules whole getFullPath() matches one of the pattern parameters will get included.
         The patterns may contain wilcards in the same syntax as in ini files.
         See cTopology::extractByModulePath() function
         example: destaddresses = moduleListByPath("**.host[*]", "**.fixhost[*]")
\item[-] moduleListByNedType("fully.qualified.ned.type",...): \\
         Returns a space-separated list of the modulenames with the given NED type(s).
         All modules whose getNedTypeName() is listed in the given parameters will get included.
         The NED type name is fully qualified.
         See cTopology::extractByNedTypeName() function
         example: destaddresses = moduleListByNedType("inet.nodes.inet.StandardHost")
\end{itemize}

The peer can be UDPSink or another UDPBasicBurst.

\subsubsection*{Bursts}

The first burst starts at \fpar{startTime}. Bursts start by immediately sending
a packet; subsequent packets are sent at \fpar{sendInterval} intervals. The
sendInterval parameter can be a random value, e.g. exponential(10ms).
A constant interval with jitter can be specified as 1s+uniform(-0.01s,0.01s)
or uniform(0.99s,1.01s). The length of the burst is controlled by the
\fpar{burstDuration} parameter. (Note that if \fpar{sendInterval} is greater than
\fpar{burstDuration}, the burst will consist of one packet only.) The time between
burst is the \fpar{sleepDuration} parameter; this can be zero (zero is not
allowed for \fpar{sendInterval}.) The zero \fpar{burstDuration} is interpreted as infinity.

\subsubsection*{Packets}

Packet length is controlled by the \fpar{messageLength} parameter.

The module adds two parameters to packets before sending:
\begin{itemize}
\item[-] sourceID: source module ID
\item[-] msgId: incremented by 1 after send any packet.
\end{itemize}
When received packet has this parameters, the module checks the order of received packets.

\subsubsection*{Operation as sink}

When \fpar{destAddresses} parameter is empty, the module receives packets and makes statistics only.

\subsubsection*{Statistics}

Statistics are collected on outgoing packets:
\begin{itemize}
\item[-] sentPk: packet object
\end{itemize}

Statistics are collected on incoming packets:
\begin{itemize}
\item[-] outOfOrderPk: statistics of out of order packets.
       The packet is out of order, when has msgId and sourceId parameters and module
       received bigger msgId from same sourceID.
\item[-] dropPk: statistics of dropped packets.
       The packet is dropped when not out-of-order packet and delay time is larger than
       delayLimit parameter. The delayLimit=0 is infinity.
\item[-] rcvdPk: statistics of not dropped, not out-of-order packets.
\item[-] endToEndDelay: end to end delay statistics of not dropped, not out-of-order packets.
\end{itemize}

%%% Local Variables:
%%% mode: latex
%%% TeX-master: "usman"
%%% End:

