\chapter{IPv6 and Mobile IPv6}
\label{cha:ipv6}


\section{Overview}
\label{sec:ipv6:overview}

Similarly to IPv4, IPv6 support is implemented by several cooperating modules.
The base protocol is in the \nedtype{Ipv6} module, which relies on the
\nedtype{Ipv6RoutingTable} to get access to the routes. Interface configuration
(address, state, timeouts, etc.) is held in the node's \nedtype{InterfaceTable}.

The \nedtype{Ipv6NeighbourDiscovery} module implements all tasks associated with
neighbour discovery and stateless address autoconfiguration. The data structures
themselves (destination cache, neighbour cache, prefix list) are kept in
\nedtype{Ipv6RoutingTable}. The rest of ICMPv6's functionality, such as error messages,
echo request/reply, etc.) is implemented in \nedtype{Icmpv6}.

Mobile IPv6 support has been contributed to INET by the xMIPv6 project.
The main module is \nedtype{xMIPv6}, which implements Fast MIPv6, Hierarchical MIPv6
and Fast Hierarchical MIPv6 (thus, $x \in {F, H, FH}$). The binding cache and
related data structures are kept in the \nedtype{BindingCache} module.


%%% Local Variables:
%%% mode: latex
%%% TeX-master: "usman"
%%% End:

