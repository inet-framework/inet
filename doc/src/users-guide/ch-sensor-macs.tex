\chapter{MAC Protocols for Wireless Sensor Networks}
\label{cha:sensor-macs}

\section{Overview}

The INET Framework contains the implementation of several MAC protocols
for wireless sensor networks (WSNs), including B-MAC, L-MAC and X-MAC.

To create a wireless node with a specific MAC protocol, use a node type 
that has a wireless interface, and set the interface type to the 
appropriate type. For example, \nedtype{WirelessHost} is a node type 
which is preconfigured to have one wireless interface, \ttt{wlan[0]}.
\ttt{wlan[0]} is of parametric type, so if you build the network from
\nedtype{WirelessHost} nodes, you can configure all of them to use
e.g. B-MAC with the following line in the ini file:

\begin{inifile}
**.wlan[0].typename = "BMacInterface"
\end{inifile}


\section{B-MAC}
\label{sec:bmac}

B-MAC (Berkeley MAC) is a carrier sense media access protocol for 
wireless sensor networks that provides a flexible interface to obtain
ultra low power operation, effective collision avoidance, and 
high channel utilization. To achieve low power operation, 
B-MAC employs an adaptive preamble sampling scheme to reduce duty cycle 
and minimize idle listening. B-MAC is designed for low traffic, 
low power communication, and is one of the most widely used 
protocols (e.g. it is part of TinyOS).

The \nedtype{BMac} module type implements the B-MAC protocol.

\nedtype{BMacInterface} is a \nedtype{WirelessInterface} with the MAC type
set to \nedtype{BMac}.


\section{L-MAC}
\label{sec:lmac}

L-MAC (Lightweight MAC) is an energy-efficient medium acces protocol designed 
for wireless sensor networks. Although the protocol uses TDMA to give nodes 
in the WSN the opportunity to communicate collision-free, the network is
self-organizing in terms of time slot assignment and synchronization. 
The protocol reduces the number of transceiver state switches and hence
the energy wasted in preamble transmissions.

The \nedtype{LMac} module type implements the L-MAC protocol, based on the
paper ``A lightweight medium access protocol (LMAC) for wireless sensor networks''
by van Hoesel and P. Havinga.

\nedtype{LMacInterface} is a \nedtype{WirelessInterface} with the MAC type
set to \nedtype{LMac}.


\section{X-MAC}
\label{sec:xmac}

X-MAC is a low-power MAC protocol for wireless sensor networks (WSNs).
In contrast to B-MAC which employs an extended preamble and preamble sampling,
X-MAC uses a shortened preamble that reduces latency at each hop and 
improves energy consumption while retaining the advantages 
of low power listening, namely low power communication, simplicity 
and a decoupling of transmitter and receiver sleep schedules.
 
The \nedtype{XMac} module type implements the X-MAC protocol, based on
the paper ``X-MAC: A Short Preamble MAC Protocol for Duty-Cycled 
Wireless Sensor Networks'' by Michael Buettner, Gary V. Yee, Eric Anderson
and Richard Han.

\nedtype{XMacInterface} is a \nedtype{WirelessInterface} with the MAC type
set to \nedtype{XMac}.


%%% Local Variables:
%%% mode: latex
%%% TeX-master: "usman"
%%% End:

