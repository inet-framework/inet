\chapter{The Power Model}
\label{cha:power}

\section{Overview}

Modeling power consumption becomes more and more important with the increasing
number of embedded devices and the upcoming Internet of things. Mobile personal
medical devices, large scale wireless environment monitoring devices, electric
vehicles, solar panels, low-power wireless sensors, etc. require paying special
attention to power consumption. The high fidelity simulation of power consumption
allows designing power sensitive routing protocols, MAC protocols, physical
layers, etc. which in turn results in more energy efficient devices.

In order to help the modeling process the power model is separated from the other
simulation models. This separation makes the model extensible and it also allows
easy experimentation with alternative implementations. In a nutshell the power
model consists of the following components:

\begin{itemize}
  \item energy consumption models
  \item energy generation models
  \item temporary energy storage models
\end{itemize}

The following sections provide a brief overview of these components. 

\section{Energy Consumer Models}

The energy consumer models describe the energy consumption of devices over time.
For example, a radio consumes energy when it transmits or receives signals, or a
CPU consumes energy when the network layer processes packets, or a display
consumes energy when it's turned on, etc. Energy consumers connect to an energy
storage that provides them with energy.

The \nedtype{StateBasedEnergyConsumer} module implements a simple radio energy
consumption model in the physical layer. This model determines the current power
consumption of the radio using constant module parameters for each valid
combination of the radio mode, the transmitter state and the receiver state.

\section{Energy Generator Models}

The energy generator models describe the energy generation of devices over time.
A solar panel, for example, produces energy based on time, the panel's position
on the globe, its orientation towards the sun and the actual weather conditions.
Energy generators connect to an energy storage that absorbs the generated energy. 

The \nedtype{AlternatingEnergyGenerator} module implements a simple model which
alternates between two modes called generation mode and sleep mode. In generation
mode it generates a randomly selected constant power for a random time interval.
In sleep mode it doesn't generate energy for another random time interval.
 
\section{Energy Storage Models}

The energy storage models describe devices that absorb energy produced by
generators, and provide energy for consumers. For example, an electrochemical
battery in a mobile phone provides energy for its display, its CPU, and its
wireless communication devices. It can also absorb energy produced by a solar
panel installed on its display, or by a portable charger plugged into the wall
socket.

The \nedtype{SimpleEnergyStorage} implements an energy storage model that
maintains a residual capacity by integrating the difference between the total
generated power and the total consumed power over time. It can initiate node
shutdown when the residual capacity drops below a configured threshold. It can
also initiate node start when the residual capacity raises above another
configured threshold. This simple model doesn't have properties such a memory
effect, self-discharge, overcharging, temperature-dependence, etc. that real
world batteries have.

%%% Local Variables:
%%% mode: latex
%%% TeX-master: "usman"
%%% End:

