\ifdraft TODO

\chapter{Cross-Layer Communication}
\label{cha:cross-layer-communication}

\section{Overview}

In the INET Framework, when an upper-layer protocol wants to send a data
packet over a lower-layer protocol, the upper-layer module just sends the
message object representing the packet to the lower-layer module, which
will in turn encapsulate it and send it. The reverse process takes place
when a lower layer protocol receives a packet and sends it up after
decapsulation.

It is often necessary to convey extra information with the packet. For
example, when an application-layer module wants to send data over TCP, some
connection identifier needs to be specified for TCP. When TCP sends a
segment over IP, IP will need a destination address and possibly other
parameters like TTL. When IP sends a datagram to an Ethernet interface for
transmission, a destination MAC address must be specified. This extra
information is attached to the message object in the form as \textit{message tags}.

Message tags are small value objects, which are attached to packets using 
the ... C++ calls...  TODO complete

 

\section{Tags}

TODO

\section{The Dispatching Mechanism}

TODO

\section{Typical Tags Understood By Various Protocols}

TODO list the tag names understood, by laters!

\fi


