\chapter{Developing Protocol Models}
\label{cha:developing-protocol-models}

\section{Overview}

This section introduces the most important modeling support features of
INET. These features facilitate the implementation of applications and
communication protocols by providing various commonly used functionality.
Thus modeling support allows rapid implementation of new models by building
on already existing APIs while the implementor can focus on the research
topics. These features differ from the reusable NED modules introduced
earlier, because they are available in the form of C++ APIs.

The easy usage of protocol services is another essential modeling support.
Applications often need to use several different protocol services
simultaneously. In order to spare the applications from using the default
OMNeT++ message passing style between modules, INET provides an easy to use
C++ socket API.

\section{Working with Packets}

\subsection{Encapsulating Packets}

TODO explain how INET packet representation supports encapsulation of packets, provide examples, etc.

\cppsnippet{PacketEncapsulationExample}{Packet encapsulation example}

\cppsnippet{PacketDecapsulationExample}{Packet decapsulation example}

\subsection{Fragmenting Packets}

explain how INET packet representation supports fragmentation of packets, provide examples, etc.

\cppsnippet{PacketFragmentationExample}{Packet fragmentation example}

\cppsnippet{PacketDefragmentationExample}{Packet defragmentation example}

\subsection{Aggregating Packets}

explain how INET packet representation supports aggregation of packets, provide examples, etc.

\cppsnippet{PacketAggregationExample}{Packet aggregation example}

\cppsnippet{PacketDisaggregationExample}{Packet disaggregation example}

\subsection{Serializing Packets}

explain how INET packet representation provides supports for serialization, provide examples, etc.

\cppsnippet{PacketSerializationExample}{Packet serialization example}

\cppsnippet{PacketDeserializationExample}{Packet deserialization example}


\subsection{Emulation Support}

explain how INET packet representation provides supports for emulation, provide examples, etc.

\cppsnippet{EmulationPacketSendingExample}{Emulation packet sending example}

\cppsnippet{EmulationPacketReceivingExample}{Emulation packet receiving example}

\subsection{Queueing Packets}

explain how INET supports queueing packet data arriving in order

\cppsnippet{PacketQueueingExample}{Packet queueing example}


\subsection{Buffering Packets}

explain how INET supports buffering packet data arriving out of order

\subsection{Reassembling Packets}

explain how INET merges packet data arriving out of order into a whole

\cppsnippet{PacketReassemblingExample}{Packet reassembling example}

\subsection{Reordering Packets}

explain how INET forms a stream from packet data arriving out of order

\cppsnippet{PacketReorderingExample}{Packet reordering example}

\subsection{Dispatching Packets}

explain how INET allows connecting multiple protocols within a network node in a very flexible way

\cppsnippet{PacketDispatchingExample}{Packet dispatching example}


\section{Resolving Addresses}

explain what kind of addresses INET provides for protocols to use: network
and MAC addresses, related protocols: ARP, DHCP, ND, etc.

address lookup by name

node lookup by MAC address

node lookup by L3 address


\section{Initializing Modules}

explain how INET supports multi stage interdependent module initialization

\cppsnippet{ModuleInitializationExample}{Module initialization example}


\section{Starting and Stopping Nodes}

explain how INET supports network node lifecycle management: startup, shutdown, crash, etc.

\cppsnippet{LifecycleOperationExample}{Lifecycle operation example}

%%%% Local Variables:
%%% mode: latex
%%% TeX-master: "usman"
%%% End:

