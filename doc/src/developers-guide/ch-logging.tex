\chapter{Logging Guidelines}
\label{cha:logging}

% TODO: Some unresolved questions:
% - what context information should be present in each log line? how do we
%   support easy and efficient searching and filtering of log output? how do we
%   support easy use of grep and similar tools?
% - do we suggest that context information that doesn't change during an event
%   should be reported only once? should context information be repeated or not
%   in subsequent lines?

This chapter describes how to write and organize log statements in the INET
framework.

\section{General Guidelines}

Log output should be valid English sentences starting with an uppercase letter
and ending with correct punctuation. Log output that spans multiple lines should
use indentation where it isn't immediately obvious that the lines are related to
each other. Dynamic content should be marked with single quotes. Key value pairs
should be labeled and separated by '='. Enumerated values should be properly
separated with spaces.

\section{Target Audience}

The people who read the log output can be divided based on the knowledge they
have regarding the protocol specification. They may know any of the following:

\begin{itemize}
	\item public interface of the protocol
 	\item internal operation of the protocol
 	\item actual implementation of the protocol
\end{itemize}

\section{Log Levels}

This section describes when to use the various log levels provided by OMNeT++.
The rules presented here extend the rules provided in the OMNeT++ documentation.

\subsection*{Fatal log level}

Target for people (not necessarily programmers) who know the public interface.
Don't report programming errors, use C++ exceptions for this purpose. Report
protocol specific unrecoverable fatal error situations. Use rarely if at all.

\subsection*{Error log level}

Target for people (not necessarily programmers) who know the public interface.
Don't report programming errors, use C++ exceptions for this purpose. Report
protocol specific recoverable error situations.

\subsection*{Warn log level}

Target for people (not necessarily programmers) who know the public interface.
Report protocol specific exceptional situations. Don't report things that occur
too often such as collisions on a radio channel.

\subsection*{Info log level}

Target for people (not necessarily programmers) who know the public interface.
Think about the module as a closed (black) box. Report protocol specific public
input, output, state changes and decisions.

\subsection*{Detail log level}

Target for users (not necessarily programmers) who know the internals. Think
about the module as an open (white) box. Report protocol specific internal state
changes and decisions. Report scheduling and processing of protocol specific
timers.

\subsection*{Debug log level}

Target for developers/maintainers who know the actual implementation.
Report implementation specific state changes and decisions. Report important
internal variable and data structure states and changes. Report current states
and transitions of state machines.

\subsection*{Trace log level}

Target for developers/maintainers who know the actual implementation.
Report the execution of initialize stages, operation stages. Report
entering/leaving functions, loops, code blocks, conditional branches, and
recursions.

\section{Log Categories}

\begin{itemize}
	\item \textit{test}: report output that is checked for in automated tests
	\item \textit{time}: report performance related data (e.g. measured wall clock
	time)
	\item \textit{parameter}: report actual parameter values picked up during
	initialization
	\item \textit{default (empty)}: report any other information using the default
	category
\end{itemize}
